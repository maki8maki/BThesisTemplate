\documentclass[../main]{subfiles}

\graphicspath{{../figures/}}

\begin{document}

\chapter{提案手法}
\label{cp:proposed_method}
\thispagestyle{empty}
\minitoc
\newpage

\section{はじめに}
\label{sec:pmethod_introduction}

本章では,提案手法について述べる.

\refsec{sec:pmethod_model}では,反射モデルについて述べる.

\refsec{sec:pmethod_method}では,分光反射率を推定する具体的な手法について述べる.

\refsec{sec:pmethod_estimation}では,分光反射率からコーン指数を推定する手法について具体的に述べる.

最後に\refsec{sec:pmethod_conclusion}では,本章のまとめを述べる.

\clearpage

\section{反射モデル}
\label{sec:pmethod_model}

\clearpage

\section{分光反射率の推定手法}
\label{sec:pmethod_method}

\clearpage

\section{コーン指数の推定手法}
\label{sec:pmethod_estimation}

\clearpage

\section{おわりに}
\label{sec:pmethod_conclusion}

本章では,提案手法について述べた.

\refsec{sec:pmethod_model}では,反射モデルについて触れ,特に本研究で採用するOren-Nayerモデルについて詳細に述べた.

\refsec{sec:pmethod_method}では,分光反射率を推定する具体的な手法について述べた.
モデルと複数の測定条件で得られた測定値の差を最小にするような$\sigma$と$\rho_{\lambda_i}$を求めることで対象の分光反射率を得る.

\refsec{sec:pmethod_estimation}では,分光反射率からコーン指数を推定する手法について述べた.

\clearpage

\end{document}
